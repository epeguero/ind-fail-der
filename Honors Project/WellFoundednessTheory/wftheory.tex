\documentclass{article}
\usepackage{bussproofs}
\usepackage{mathabx}

\begin{document}

\section{Introduction}

To explore the relationship between the Law of
Excluded Middle (LEM) and Induction, we want to examine whether the former can
replace latter and vice versa in a proof. To this end, we model proofs
as mathematical objects (i.e., from a metamathematical standpoint) by
employing technical devices from proof theory. Specifically, we
examine a constructive first-order sequent calculus with an induction rule and LEM. This avenue of exploration leverages the proof-theoretic results of soundness, completeness, and others, allowing for the possible future use of
automated theorem provers.

This document is structured as follows:
\begin{itemize}
\item Section 2 provides the proof theoretic language that will frame our approach
\item Section 3 presents future ideas for examining LEM and Induction
\end{itemize}


\section{Metamathematical Approach}

In Kleene's \textit{Introduction to Metamathematics}, the machinery of metamathematics is described by a three layered architecture:
\begin{enumerate}
\item A collection of informally understood truths composing a \textit{theory of interest}.
\item The abstracted or \textit{formalized theory} that symbolically codifies statements from the informal theory. This is the object of study in metamathematics.
\item Informal definition and characterization of the formal theory, i.e. the \textit{metatheory}.
\end{enumerate}

The metatheory, (3), must make use of purely effective (i.e., algorithmic, constructive, finitary) means to describe and make claims about the formal theory, (2), without referring to the underlying theory, (1), as an interpretation for the symbols. (2) itself is the object of study in proof theory, not the interpretation motivated by (1). In what follows, our metamathematical approach is dissected with this architectural scheme in mind (though not presented in the same order).

\subsection{Formalization: First-order Theory}
A formal theory is a set of strings or \textit{theorems}, each representing a statement in the underlying informal theory. The \textit{syntax} of a formal theory determines the membership of a theorem in the theory and is defined by the following components:
\begin{itemize}
\item Language: a set of available symbols from which strings are constructible.
\item Grammar: a deductive system that specifies how theorems are generated.
\end{itemize} 

\subsubsection{Language: First-order Symbols}
The language of a first-order theory is defined by logical symbols and non-logical symbols each representing logical or theory specific constructs, respectively. These symbols are defined as follows:

\begin{itemize}
\item Logical Symbols
\begin{itemize}
\item Logical Connectives: $\wedge, \vee, \lnot, \to, \forall, \exists$
\item Equality: the binary relation $=$
\item Variables: a denumerable set of symbols $x_0, x_1, ..., x_n, ...$
\end{itemize}

\item Non-logical Symbols
\begin{itemize}
\item Constants: a denumerable set of symbols $c_0, c_1, ..., c_n, ...$
\item Function Symbols: zero or more symbols $f^{n_1}_1, f^{n_2}_2, ..., f^{n_k}_k, ...$. The superscript denotes the number of arguments the function takes, i.e. its \textit{arity}, whereas the subscript identifies the function symbol.
\item Predicate Symbols: zero or more symbols $P^{n_1}_1, P^{n_2}_2, ..., P^{n_k}_k, ...$. The symbolic naming scheme is the same as that for functions.
\end{itemize}

\end{itemize}

These symbols are used to generate the nouns, or \textit{terms}, of the theory. Terms are defined recursively as the smallest set of strings that satisfy the following criteria:
\begin{itemize}
\item Each variable and constant is a term.
\item If $t_1, t_2, ... t_k$ are terms, then $f^{n_k}_k t_1 t_2 ... t_k$ is a term.
\end{itemize}

Theorems represent statements made about terms and are recursively defined as the smallest set of strings that satisfies the following criteria:
\begin{itemize}
\item If $t_1$ and $t_2$ are terms, then $t_1 = t_2$ is a theorem.
\item If $t_1, t_2, ... t_k$ are terms, then $P^{n_k}_k t_1 t_2 ... t_k$ is a theorem.
\item If $\theta_1$ and $\theta_2$ are theorem's, then $\theta_1 \wedge \theta2$ is a theorem.
\item If $\theta$ is a theorem and $x$ is a variable, then $\exists x \theta$ is a theorem.
\item ...similar criteria for the other logical connectives
\end{itemize}

\subsubsection{Grammar: Deductive System}
A deductive system defines the criteria for membership of a theorem in the theory and is composed of a set of inference rules and two sets of axioms, $\Lambda$ and $\Gamma$, denoting the logical and non-logical axioms respectively. This distinction serves a similar role to that of the symbols in the language, namely, it enables a discussion of different classes of logic (e.g., classical, intuitionistic, etc...) separate from that of the mathematical theory itself. 

In intuitionistic sequent calculus, the rules of inference dictate how sequents may be constructed from previous sequents; the axioms provide sequents axiomatically. A set of theorem, $\Gamma$ and a single theorem, $\theta$, form a sequent if they are related by the \textit{syntactic entailment} relation, $\dashv$. $\Gamma \dashv \theta$ holds if and only if there is a finite tree of sequents formed by the deductive system.

\subsubsection{Metatheory}
Ideally, a first-order theory faithfully represents the underlying theory. In what follows, we examine this expectation.

\subsubsection{Theorem Interpretation}
An interpretation of a theorem consists of the selection of a model and the denotation of terms. A model consists of a \textit{universe} or \textit{domain of discourse} and an \textit{interpretation function} that maps each constant symbol to an element of the universe, each function symbol to an n-ary function with codomain in the universe, and each predicate as an n-ary relation in the universe. This model is used to recursively provide a \textit{denotation} of each term in the theorem. The logical symbols are given their usual meaning. An interesting case in truth determination arises for quantified theorems, wherein the quantifier is removed and substitution is performed for all elements of the universe; truth is required for every substitution in the case of universal quantification and only for one substitution in the case of existential quantification. A theorem is said to be \textit{valid} if it holds under all possible interpretations for which the axioms are true, expressed as the \textit{semantic entailment} relation $\Gamma \vDash \theta$ (the logical axioms are assumed to hold implicitly).

\subsubsection{Properties of First-order Logic}
Intuitively, we would like each theorem to be valid. For example, we expect our theory of well-founded relations to hold under every interpretation of $\prec$ and $P$ that satisfies our induction axioms. As another example, we would expect group theory to include exactly those theorem's that are true in every interpretation that satisfies the axioms of group theory. 

A deductive system is \textit{sound} if every provable theorem is valid ($\Gamma \dashv \theta \to \Gamma \vDash \theta$), and it's \textit{semantically complete} if every valid theorem is provable ($\Gamma \vDash \theta \to \Gamma \dashv \theta$). An attractive feature of first-order logic is that it is both sound and semantically complete, whereas the latter property does not hold for higher-order logics. 

It is worth commenting on the notion of \textit{syntactically complete} axioms. A set of non-logical axioms, $\Gamma$, is syntactically complete if for any theorem, $\theta$, either $\Gamma \dashv \theta $ or $\Gamma \dashv \lnot \theta $. Syntactic completeness implies that all interpretations of the theory for which the $\Gamma$ holds agree on the truth value of every sentence, implying the presence of a unique model. Furthermore, decidability follows from syntactic completeness, since every sentence or its negation is derivable. Unfortunately, \textit{Godel's First Incompleteness Theorem} states that if $\Gamma$ contains "enough of the theory of the natural numbers" (enough to cleverly encode the statement, "this statement is unprovable from the axioms"), then syntactic completeness implies that the theory is unsound. An example of a violation of syntactic completeness is given by the theorem $\forall x (x + 1 > x)$ in Peano Arithmetic: the theorem is true in the standard interpretation of the natural numbers, but not in the interpretation given by the universe ${0,1}$ with addition modulo 2.

\subsection{Formalizing the Theory of Interest: Well-founded Relations}
The current examination focuses on the theory of objects for which inductive proofs apply, motivated by the specific case of induction on derivation trees. As derivation trees are susceptible to structural induction, we formalize the idea of inductive objects by those constructed through a well-founded relation; thus, our underlying theory is that of well-founded relations. 
We will attempt to avoid set theory (that is, the set inclusion operator), instead focusing on a first-order language consisting solely of the binary predicate $\prec$ and the unary predicate $P$. We will encode well-foundedness using the following non-logical axioms:
$$\forall x (P(x) \to \exists y (P(y) \wedge \forall z (P(z) \to y \prec z))) $$
$$\forall x \forall y ((y \prec x \to P(y)) \to P(x)) \to \forall x P(x)$$
$$\forall x (\lnot (x \prec c_i))$$ (here we add infinite constants, representing base cases)
(The second rule may follow from the first, depending on the available logical axioms)

Further, we will select logical axioms corresponding to an intuitionistic logic (intro-elim forms) with LEM and add axioms as needed.

\section{Future Work}
A strategy for replacing induction with reductio ad absurdum (RAA), a corrollary of LEM is outlined as follows: 
\begin{enumerate}
\item We have $\forall x \forall y ((y \prec x \to P(y)) \to P(x))$ from the antecedent of Modus Ponens with the induction axiom.
\item We have $\forall x (P(x) \to \exists y (P(y) \wedge \forall z (P(z) \to y \prec z))) $ as a non-logical axiom. Thus, there is a smallest element for which the predicate holds.
\item NEED TO FINISH
\end{enumerate}


\end{document}