\documentclass[titlepage]{article}
\usepackage[utf8]{inputenc}
\usepackage{amsmath}
\usepackage{amsfonts}
\usepackage{amssymb}
\usepackage{bussproofs}
\usepackage{graphicx}
\usepackage{apacite}
\newenvironment{bprooftree}
  {\leavevmode\hbox\bgroup}
  {\DisplayProof\egroup}
	
\begin{document}

%%%%%%%%%%%%%%%%%%%%%%%%%%%%%%%%%%%%%%%%%
% University Assignment Title Page 
% LaTeX Template
% Version 1.0 (27/12/12)
%
% This template has been downloaded from:
% http://www.LaTeXTemplates.com
%
% Original author:
% WikiBooks (http://en.wikibooks.org/wiki/LaTeX/Title_Creation)
%
% License:
% CC BY-NC-SA 3.0 (http://creativecommons.org/licenses/by-nc-sa/3.0/)
% 
% Instructions for using this template:
% This title page is capable of being compiled as is. This is not useful for 
% including it in another document. To do this, you have two options: 
%
% 1) Copy/paste everything between \begin{document} and \end{document} 
% starting at \begin{titlepage} and paste this into another LaTeX file where you 
% want your title page.
% OR
% 2) Remove everything outside the \begin{titlepage} and \end{titlepage} and 
% move this file to the same directory as the LaTeX file you wish to add it to. 
% Then add \input{./title_page_1.tex} to your LaTeX file where you want your
% title page.
%
%%%%%%%%%%%%%%%%%%%%%%%%%%%%%%%%%%%%%%%%%

%----------------------------------------------------------------------------------------
%	PACKAGES AND OTHER DOCUMENT CONFIGURATIONS
%----------------------------------------------------------------------------------------

\begin{titlepage}

\newcommand{\HRule}{\rule{\linewidth}{0.5mm}} % Defines a new command for the horizontal lines, change thickness here

\center % Center everything on the page
 
%----------------------------------------------------------------------------------------
%	HEADING SECTIONS
%----------------------------------------------------------------------------------------

\textsc{\LARGE University of South Florida}\\[1.5cm] % Name of your university/college
\textsc{\Large Honors Thesis}\\[0.5cm] % Major heading such as course name

%----------------------------------------------------------------------------------------
%	TITLE SECTION
%----------------------------------------------------------------------------------------

\HRule \\[0.4cm]
{ \Large \bfseries An Examination of Induction on Failing Derivations in Relation to Coinduction and the Law of Excluded Middle}\\[0.4cm] % Title of your document
\HRule \\[1.5cm]
 
%----------------------------------------------------------------------------------------
%	AUTHOR SECTION
%----------------------------------------------------------------------------------------

\begin{minipage}{0.4\textwidth}
\begin{flushleft} \large
\emph{Author:}\\
Edwin \textsc{Peguero}\\ 
U89344461 \\ % Your name
epeguero@mail.usf.edu

\end{flushleft}
\end{minipage}
~
\begin{minipage}{0.4\textwidth}
\begin{flushright} \large
\emph{Thesis Chair:} \\
Dr. Jay \textsc{Ligatti}\\ ~ \\ ~ \\ % Supervisor's Name
\end{flushright}
\end{minipage}\\[4cm]

% If you don't want a supervisor, uncomment the two lines below and remove the section above
%\Large \emph{Author:}\\
%John \textsc{Smith}\\[3cm] % Your name

%----------------------------------------------------------------------------------------
%	DATE SECTION
%----------------------------------------------------------------------------------------

%{\large \today}\\[3cm] % Date, change the \today to a set date if you want to be precise

%----------------------------------------------------------------------------------------
%	LOGO SECTION
%----------------------------------------------------------------------------------------

%\includegraphics{Logo}\\[1cm] % Include a department/university logo - this will require the graphicx package
 
%----------------------------------------------------------------------------------------

\vfill % Fill the rest of the page with whitespace

\end{titlepage}

\section{Introduction}

Significant research in mathematics and computer science focuses on \textbf{recursive objects}, which, informally, are defined by a collection of recipes through which objects can be built out of sub-objects of the same kind. Examples of recursive objects include lists, programs, and the natural numbers, which can be built out of sub-lists, sub-programs, and predecessors respectively. For many recursive objects, it is desirable to demonstrate that a property holds over all \textit{finitely constructible} objects. This set is called the \textbf{inductive set}, a subset of the so-called \textbf{coinductive set} that further includes objects with \textit{infinite constructions}. For example, the inductive set of programs in some programming languages can be proven to have the property of \textit{type-safety} (i.e., to never get stuck during execution). Such proofs, called \textbf{inductive proofs}, proceed by demonstrating that for each way of constructing a recursive object, the property of interest holds on the object whenever it holds on all of its sub-objects (an assumption known as the \textit{inductive hypothesis}). 

An inductive set is often specified by a \textbf{deductive system}. In these systems, objects are represented by \textbf{judgements} (the generic form of which is called the \textit{judgement form}) and each construction recipe by a labeled \textbf{inference rule}. Each inference rule is stated as zero (in the case of \textit{axioms}) or more \textit{premises} and one \textit{conclusion}, expressed in terms of quantified meta-variables. A judgement is said to be \textbf{valid} whenever it is the conclusion of a rule whose premises are valid. Thus, the inductive set is represented by the set of valid judgements.

\begin{figure}[h]
\caption{Deductive System: Natural Numbers}
\label{fig:nat-ded}

\begin{center}
  \boxed{n \in \mathbb{N}}
\end{center}

\[
\begin{bprooftree}
\AxiomC{}
\RightLabel{\scriptsize 0 Rule}
\UnaryInfC{$0 \in \mathbb{N}$}
\end{bprooftree}
\begin{bprooftree}
\AxiomC{$n \in \mathbb{N}$}
\RightLabel{\scriptsize Successor Rule}
\UnaryInfC{$n+1 \in \mathbb{N}$}
\end{bprooftree}
\]
\end{figure}

Figure~\ref{fig:nat-ded} shows the specification of the natural numbers via a deductive system. The inference rules can be used to construct arbitrary elements of the natural numbers; for example, Figure 2 shows how the number $2 = 0+1+1$ can be built. As each natural number is a recursive object with a finite construction, the set of natural numbers can be regarded as an inductive set.

\begin{figure}[h]
  \caption{Derivation Tree for $2 \in \mathbb{N}$}
  \begin{prooftree}
	\AxiomC{$0 \in \mathbb{N}$}
	  \RightLabel{\scriptsize 0 Rule)}
	\UnaryInfC{$0 + 1 \in \mathbb{N}$}
	  \RightLabel{\scriptsize Successor Rule}
	\UnaryInfC{$0 + 1 + 1 \in \mathbb{N}$}
  \end{prooftree}
\end{figure}

In theoretical computer science, the \textit{dynamic semantics} of a programming language, that is, the manner in which programs execute, is often defined by a deductive system. Figure~\ref{fig:lam-step} defines the \textit{small step} semantics of the \textit{$\lambda$-calculus}. Expressions in the $\lambda$-calculus can be constructed in the following ways:
\begin{itemize}
  \item $x$: $x \in V$, where $V$ is a set of variables
  \item $\lambda x.e$: a function with parameter $x$ constructed recursively by $x \in V$ and $e \in E$
  \item $e~e'$: function application ($e'$ is a parameter for function $e$) constructed recursively by $e, e' \in E$
\end{itemize} 

\begin{figure}[h]
	\caption{Deductive System: Dynamic Semantics of Call-by-Name $\lambda$-Calculus}
	\label{fig:lam-step}

	\begin{center}
	\boxed{e \Rightarrow e'}
	\end{center}
	
	\[
	\begin{bprooftree}
	\AxiomC{$e \Rightarrow e'$}
	\RightLabel{\scriptsize Search Rule }
	\UnaryInfC{$e~e''\Rightarrow e~e''$}
	\end{bprooftree}
	\begin{bprooftree}
	\AxiomC{}
	\RightLabel{\scriptsize $\beta$-step Rule}
	\UnaryInfC{$\lambda x.e ~ e' \Rightarrow [e/x]e'$}
	\end{bprooftree}
	\]

\end{figure}

The boxed judgement form in Figure~\ref{fig:lam-step} shows that the stepping judgement ($\Rightarrow$) relates two expressions in the $\lambda$-calculus. $[e/x]e'$ denotes a careful (more formally, \textit{capture-avoiding}) substitution of $e$ for the free variables $x$ in $e'$. Using the inference rules, we can show how the program $(\lambda x.x ~ \lambda x.(x~x)) ~ \lambda x.x$ takes a step:

\begin{center}
\begin{bprooftree}

  \AxiomC{}
  \RightLabel{\scriptsize $\beta$-step Rule}
  \UnaryInfC{$\lambda x.x ~ \lambda x.(x~x) \Rightarrow \lambda x.(x~x)$}
  \RightLabel{\scriptsize Search Rule}
  \UnaryInfC{$(\lambda x.x ~ \lambda x.(x~x)) ~ \lambda x.x \Rightarrow \lambda x.(x~x) ~ \lambda x.x$}
\end{bprooftree}
\end{center}

As these example demonstrate, every valid judgement corresponds to a \textbf{derivation tree} formed by a sequence of deductions from premises to conclusions. Thus, for every valid judgement, there exists at least one derivation tree. This fact is used in the inductive technique called \textbf{proof on derivations}, whereby the inductive proof is performed on the inductive set of derivation trees. Thus, the proof is executed by demonstrating that for every way of building a derivation tree (i.e., inference rule), the property of interest holds whenever it holds on all sub-trees. Intuitively, the validity of this technique lies in the fact that each derivation tree represents the construction of a recursive object in the original inductive set.

Proofs on derivations are examples of \textbf{intuitionistic} proofs, an important class of proofs that are especially valuable in computer science theory. Such proofs deny the use of the \textbf{law of excluded middle}, a law in classical logic that states that every proposition is either true or false: there is no alternative "middle" interpretation. In classical logic, this law enables \textit{proofs by contradiction}, wherein if the negation of a proposition is shown to be false by means of a contradiction, then the truth of the original proposition follows by the law of excluded middle. In proofs by contradiction, no evidence for the truth of a proposition is needed, but rather for the falsity of its negation. On the other hand, intuitionistic proofs are \textit{constructive} in that they present direct evidence for a proposition. For this reason, intuitionistic proofs often lead directly to useful algorithms.

Oftentimes, it is necessary to show that some property holds on the set of \textit{finitely failing} judgements. \textbf{Proof on failing derivations}, due to Ligatti, is a new technique developed for this purpose. The technique proceeds by demonstrating that for each rule for constructing a derivation tree and for each way that this construction can fail, if the property holds on the failed sub-tree, then it holds on the tree. The similarity of proof on derivations and proof on failing derivations presents two interesting questions: can the technique be considered formally constructive? Is the duality of these proof techniques related to the duality between inductive and coinductive sets? It is the goal of this thesis to establish the constructive formalization of these proof techniques and explore a potential link to coinduction.

\section{Background and Techniques}
We examine the possibility of a constructive, set-theoretic formalization for proof on failed derivations by considering how such a formalization is obtained for conventional inductive proofs. If a set of judgements is regarded as a relation on sets (that is, a subset of $X = X_1 \times X_2 \times .... \times X_n$) then the inductive set is the relation containing exactly all the valid judgements with finite constructions. Inference rules, from which valid judgements can be constructed, are encoded as a set $R \subseteq \mathcal{P}(X) \times X$ of \textit{ground rules}. Each ground rule $(S,x) \in \mathcal{P}(X) \times X$ represents a substitution for the meta-variables in an inference rule: $S$ is the set of premises from which a conclusion $x$ follows \cite{Sangiorgi2011}. 

Intuitively, the inductive set can be constructed by the following algorithm: starting with $T = \emptyset$, we add to $T$ all judgements derivable from $R$ using premises in $T$. In the first iteration only the axiomatically derivable judgements are added to $T$; more complex derivations occur in subsequent iterations. This algorithm can be described as the iterative application of the operator $\Phi_R:\mathcal{P}(X) \mapsto \mathcal{P}(X)$, called the \textbf{rule functional}, that takes a set of judgements, $T$ and produces the set of valid judgements under $R$ using the judgements in $T$ as premises. $$\Phi_R(T) = \{\,x \in X \,|\,\exists (S,x) \in R : S \subseteq T \,\}$$

In this framework, the dynamic semantics of the lambda calculus as shown in Figure \ref{fig:lam-step} can be specified by a binary relation on $E \times E$, where $E$ is itself the inductive set of legal expressions. The corresponding ground rules are then given by: 
$$R = \{(\emptyset ,(\lambda x.e \, e',[e/x]e')) \,|\, \forall e,e' \in E, \forall x \in V\} \,\cup\, \{(\{(e,e')\},(e\,e'',e'\,e'')) \,|\, \forall e,e',e'' \in E\}$$ 
Subsequently, the rule functional induced by $R$ is given by: 
$$\Phi_R(T) = \{(\lambda x.e \, e',[e/x]e') \,|\, \forall e,e' \in E, \forall x \in V \} \,\cup\, \{(e\,e'',e'\,e'') \,|\, \forall (e,e') \in T, \forall e'' \in E\} $$

Similarly, the deductive system in Figure~\ref{fig:nat-ded} that defines the natural numbers induces the following rule functional: 
$$\Phi_R(T) = \{0\} \,\cup\, \{N + 1 \,|\, \forall N \in T\} $$
In this case, it's easy to see how the iterative application of the rule functional constructs the inductive set:
$$\Phi_R(\emptyset) = \{0\}$$
$$\Phi_R(\Phi_R(\emptyset)) = \{0,0 + 1\}$$
$$...$$
$$\Phi_R^n(\emptyset) = \{0,0 + 1, ..., 0 + 1 + ... + 1 \text{ (n-1 1's)}\} [\forall n \in \mathbb{N}]$$

The domain of the rule functional, $\mathcal{P}(X)$, equipped with the set-inclusion relation forms a \textit{complete lattice}. A \textit{partially ordered set} $L$ (i.e.,A set equipped with an antisymmetric, reflexive, and transitive relation) is a complete lattice iff $\forall S \subseteq L: \vee S \in \L, \wedge S \in L$, where $\vee S$ is the \textit{join} (least upper bound) and $\wedge S$ is the \textit{meet} (greatest lower bound) of $S$, respectively. Further, $\mathcal{P}(X) \in T$ is a \textit{pre-fixed point} iff $T \subseteq \Phi_R(T)$, a \textit{post-fixed point} iff $T \subseteq \Phi_R(T)$, and a \textit{fixed point} iff it is both a pre-fixed and post-fixed point. With this language, we can describe the construction of the inductive set as starting at $\emptyset \in \mathcal{P}(X)$ and climbing the powerset lattice, $\mathcal{P}(X)$, traversing from post-fixed point to post-fixed point, until the least fixed point (denoted by $lfp(F)$) is reached: the first post-fixed point that is also a pre-fixed point. This motivates the definition of an \textbf{inductive set on a complete lattice, $L$, with respect to $F:L \mapsto L$}: 

$$F_{ind} = \wedge \{\, T \in \mathcal{P}(X) \,|\, T \subseteq \Phi_R(T) \,\}$$ 

Figure~\ref{ind-construction} demonstrates the construction of the natural numbers as the iterative application of the rule functional on the empty set. The existence of the inductive set is guaranteed by \textbf{Tarski’s Fixed Point Theorem}, which states that given a complete lattice $L$ and function $F : L \mapsto L$, if $F$ is monotone, i.e. $\forall A,B \in \mathcal{P}(X) : A \subseteq B \implies F(A) \subseteq F(B)$, then $F_{ind} = lfp(L)$. As the rule functional is indeed monotone, the existence of the inductive set follows from the theorem.

\begin{figure}[h]
\caption{Construction of the Inductive Set as Traversal of Post-Fixed Points}
\label{ind-construction}

$$\emptyset \subseteq \Phi_R(\emptyset) \subseteq \Phi_R(\Phi_R(\emptyset)) \subseteq ... \subseteq \Phi_R^n(\emptyset) \subseteq ... \subseteq \mathbb{N} \supseteq \Phi_R(\mathbb{N})$$

\end{figure}

An inductive proof that some property $Pr$ holds on the inductive set is tantamount to showing that, letting $T = \{\, x \in X \,|\, Pr(x)\}$, $F_{ind} \subseteq T$. This can be accomplished by showing $\Phi_R(T) \subseteq T$ (i.e., $T$ is a pre-fixed point w.r.t $\Phi_R$), since by definition, $F_{ind}$ is a lower bound of (i.e., a subset of) all pre-fixed points. In the context of a rule functional, pre-fixed points are said to be \textit{closed forwards under the rules}. Demonstrating forward closure of a set coincides with the familiar structure of inductive proofs: $\forall x \in \Phi_R(T)$ (\textit{for each way of concluding a valid judgement, assuming the premises satisfy the property}) it follows that $x \in T$ (\textit{the valid judgement also satisfies the property}).

As Tarski's Fixed Point Theorem can be proven intuitionistically, it follows that inductive proofs are constructive. It remains to be seen whether proof on failing derivations can be constructively formalized in this manner.

\section{Work Schedule}
The tasks that are to be completed include:
\begin{itemize}
  \item A selection of axiomatic system under which to fully formalize proof on failing derivations
  \item A formal justification of the proof technique on failing derivations 
  \item An exploration of coinduction and its relationship with proof on failing derivations
\end{itemize}

Each task is completed once it is clearly, rigorously, and correctly explained in a written document. Therefore, several drafts of the thesis will be examined by the thesis chair: at least one for each task. All tasks are expected to be completed sometime by late October, after which the final draft will be developed and a presentation assembled.\\

Word Count: 1515

\bibliographystyle{apacite}
\bibliography{induction}{}


\end{document}