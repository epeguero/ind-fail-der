\section{Proof by Induction on Failing Derivations}
\label{sec-failing}
A proof by induction on failing derivations proves $\forall x \in X: P(x) \Rightarrow x \in F_{in}$ by proving the contrapositive of $\forall x \in X: x \in (F_{in})^C \Rightarrow \lnot P(x)$ by induction, where `$^C$' denotes the set complement operation.

\begin{itemize}
\item \textsc{Proof by Induction on Failing Derivations}:
$$[\forall x \in X: x \in (F_{in})^C \Rightarrow \lnot P(x)] \Rightarrow [\forall x \in X: P(x) \Rightarrow x \in F_{in}]$$
\end{itemize}

The technique therefore relies on both a \textit{proof by contrapositive} and on a proof by induction. In general, the former proof technique depends on a meta-theoretical application of the LEM, which arises from the fact that a proof by contrapositive demonstrates $P \Rightarrow \lnot \lnot Q$, but $\lnot\lnot Q \Rightarrow Q$ (i.e., the principle of \textit{double negation}) is sufficient to prove LEM. We can remove this dependence by proving $\forall x \in X: P(x) \vee \lnot P(x)$ for the particular predicate, $P$. 

On the other hand, the latter proof technique requires that $(F_{in})^C$ be an inductive set. Intuitively, $(F_{in})^C$ is inductive when the derivation of each $C \not\in F_{in}$ fails finitely. In such cases, we expect that there should be a correspondingly finite refutation. In general, however, a derivation may fail infinitely; in such cases, we do not expect for $(F_{in})^C$ to be inductive. We capture this intuition in the context of a monotonic rule functional, $\Phi_R$, by defining a complement rule functional, $\Phi_{R'}$, induced by the ground rules of the complement deductive system, $R'$, and claim that $lfp(\Phi_R)^C = gfp(\Phi_{R'})$. We conjecture that $lfp(\Phi_R)^C$ is inductive, and therefore that proof by induction on failing derivations is a sound proof technique, iff $lfp(\Phi_{R'}) = lfp(\Phi_{R'})$.

In addition, we observe that, in the case where $F_{in} = F_{co}$, proofs by induction on failing derivations may replace proofs by rule co-induction, since in this case, the proof goal of both techniques coincide.

