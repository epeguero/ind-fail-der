\documentclass[article]{journal}
\usepackage{amsmath,amssymb}
\usepackage{bussproofs}
\newcommand{\justif}[2]{&{#1}&\text{#2}}

\begin{document}

We explore induction in a classical predicate sequent calculus. That is, the derivability relation between a multiset of sequents, $\Gamma$ and an endsequent $A$ will be formally represented as $\Gamma \Rightarrow A$. At the meta-deductive level, we shall denote derivability between a set of non-logical axioms, $\Sigma$, and a formal sequent, $\Gamma \Rightarrow A$, using the traditional turnstile symbol: $\Sigma \vdash \Gamma \Rightarrow A$. 

In what follows, we make use of the following non-logical axioms (i.e., $\Sigma = {\textbf{P1}, \textbf{P2}}$), derivable in Peano arithmetic:
\begin{alignat*}{1}
&\textbf{P1}: \Gamma \Rightarrow \forall n: (n=0 \vee \exists m: n = S(m)) \\
&\textbf{P2}: \Gamma \Rightarrow \forall n: \forall m: (n = S(m) \to m < n)
\end{alignat*}

In addition to $\Sigma$, we will have the logical axiom for stating assumptions $LA: \Gamma, A \Rightarrow A$, the standard \textit{left} and \textit{right} rules of inference for sequent calculus, the \textit{rule of excluded middle} (EM), reductio ad absurdum (RAA), and a rule for the existence of an \textit{infimum} for each predicate, $A$: \\

\vskip 0.2in
\AxiomC{}
\RightLabel{\scriptsize(L$\bot$)}
\UnaryInfC{$\Gamma, \bot \Rightarrow A$}
\DisplayProof
\vskip 0.2in

\AxiomC{$\Gamma \Rightarrow A(m/n)$}
\RightLabel{\scriptsize(R$\forall$)}
\UnaryInfC{$\Gamma \Rightarrow \forall n:A$}
\DisplayProof \hskip 0.5in
\AxiomC{$A(t/n), \forall n:A, \Gamma \Rightarrow C$}
\RightLabel{\scriptsize(L$\forall$)}
\UnaryInfC{$\forall n:A, \Gamma \Rightarrow C$}
\DisplayProof 

\vskip 0.2in
\AxiomC{$\Gamma \Rightarrow A(t/n)$}
\RightLabel{\scriptsize(R$\exists$)}
\UnaryInfC{$\Gamma \Rightarrow \exists n:A$}
\DisplayProof \hskip 0.5in
\AxiomC{$A(m/n), \Gamma \Rightarrow C$}
\RightLabel{\scriptsize(L$\exists$)}
\UnaryInfC{$\exists n:A, \Gamma \Rightarrow C$}
\DisplayProof

\vskip 0.2in
\AxiomC{$\Gamma, A \Rightarrow C$}
\RightLabel{\scriptsize(R$\to$)}
\UnaryInfC{$\Gamma \Rightarrow A \to C$}
\DisplayProof \hskip 0.5in
\AxiomC{$\Gamma \Rightarrow A$}
\AxiomC{$\Gamma, B \Rightarrow C$}
\RightLabel{\scriptsize(L$\to$)}
\BinaryInfC{$A \to B, \Gamma \Rightarrow C$}
\DisplayProof

\vskip 0.2in
\AxiomC{$\Gamma \Rightarrow A$}
\AxiomC{$\Gamma \Rightarrow B$}
\RightLabel{\scriptsize(R$\wedge$)}
\BinaryInfC{$\Gamma \Rightarrow A \wedge B$}
\DisplayProof \hskip 0.5in
\AxiomC{$\Gamma, A, B \Rightarrow C$}
\RightLabel{\scriptsize(L$\wedge$)}
\UnaryInfC{$\Gamma, A \wedge B \Rightarrow C$}
\DisplayProof

\vskip 0.2in
\AxiomC{$\Gamma \Rightarrow A$}
\RightLabel{\scriptsize(R$\vee$)}
\UnaryInfC{$\Gamma \Rightarrow A \vee B$}
\DisplayProof \hskip 0.5in
\AxiomC{$\Gamma, A \Rightarrow C$}
\AxiomC{$\Gamma, B \Rightarrow C$}
\RightLabel{\scriptsize(L$\vee$)}
\BinaryInfC{$\Gamma, A \vee B \Rightarrow C$}
\DisplayProof

\vskip 0.2in
\AxiomC{$\Gamma, A \Rightarrow C$}
\AxiomC{$\Gamma, \lnot A \Rightarrow C$}
\RightLabel{\scriptsize(EM)}
\BinaryInfC{$\Gamma \Rightarrow C$}
\DisplayProof \hskip 0.5in

\vskip 0.2in
\AxiomC{$\Gamma \Rightarrow \exists n:A$}
\RightLabel{\scriptsize(Infimum)}
\UnaryInfC{$\Gamma \Rightarrow \exists n:(A \wedge \forall m:(m < n \to \lnot A(m/n)))$}
\DisplayProof \\ \vskip 0.2in

In the rules above, $A(t/n)$ represents a substitution for terms, where $t$ is free for $n$ in $A$. Further, we have the restriction for Rule R$\forall$ that $m$ cannot be free in either $\Gamma$. In Rule L$\exists$, we have a similar restriction: $m$ cannot be free in either $\Gamma$, $\exists n:A$, or $C$.

We will also refer to derivable theorems, such as DeMorgan's Theorem for quantifiers (quantifier negation) and Reductio Ad Absurdum (RAA). We will assume that weakening and contraction lemmas are also rules of the deductive system.

In Theorem 1, we prove, via a derivation, the admissibility of induction for an arbitrary predicate, $P$. Theorem 2 shows the admissibility of (EM) in the sequent logic devoid of either EM or RAA: \\
\noindent \newline
\textsc{Theorem 1:} If $\Sigma \vdash \Gamma \Rightarrow P(0)$ and $\Sigma \vdash \Gamma \Rightarrow \forall n: P(n) \to P(S(n))$ then $\Sigma \vdash \Gamma \Rightarrow \forall n: P(n)$ \\
\noindent \newline
\textsc{Theorem 2:} If $\Sigma \vdash \Gamma, A \Rightarrow C$ and $\Sigma \vdash \Gamma, \lnot A \Rightarrow C$ then $\Sigma \vdash \Gamma \Rightarrow C$, for arbitrary $A$ \\


Proof of Theorem 1:
\begin{alignat*}{2}
&1) \Delta \vdash \exists n: \sigma(n) \wedge \forall m: m < n \to \lnot\lnot\sigma(m)		\justif{\quad}{Rule: Inf}\\
&2) \Delta' \vdash N=0 \vee \exists m: N = S(m)		\justif{\quad}{PC1}\\
&3) \Delta', N=0 \vdash \lnot sigma(0)				\justif{\quad}{Rule: = E}\\
&4) \Delta', N=0 \vdash \bot                			\justif{\quad}{3), Rule: Non-contradiction}\\
&5) \Delta' \vdash \lnot N=0							\justif{\quad}{4), Rule: Reductio ad absurdum}\\
&6) \Delta', N=S(M) \vdash M < N						\justif{\quad}{PC2}\\  
&7) \Delta', N=S(M) \vdash \lnot\lnot\sigma(M) 		\justif{\quad}{6), Rule: $\to$ E}\\
&8) \Delta', N=S(M) \vdash \sigma(M) 				\justif{\quad}{7), Double Negation}\\
&9) \Delta', N=S(M) \vdash \sigma(S(M)) 				\justif{\quad}{8), Rule: $\to$ E}\\
&10) \Delta', N=S(M) \vdash \sigma(N) 				\justif{\quad}{9), Rule: = E}\\
&11) \Delta', N=S(M) \vdash \bot 					\justif{\quad}{10), Rule: Non-contradiction}\\
&12) \Delta' \vdash \lnot N=S(M) 					\justif{\quad}{11), Rule: Reductio ad absurdum}\\
&13) \Delta' \vdash \lnot\exists m: N=S(m) 			\justif{\quad}{12), Rule: $\exists$ I}\\
&14) \Delta' \vdash \lnot N=0 \wedge \lnot\exists m: N=S(m) 		\justif{\quad}{5), 13), Rule: $\wedge$ I}\\
&15) \Delta' \vdash \lnot (N=0 \vee \exists m: N=S(m)) 			\justif{\quad}{14), DeMorgan's Theorem}\\
&16) \Delta' \vdash \bot 							\justif{\quad}{2), 15), Rule: Non-contradiction}\\
&17) \Delta \vdash \bot 								\justif{\quad}{16), 1), Rule: $\exists$ E}\\
&18) \Gamma, \sigma(0), \forall n: \sigma(n) \to \sigma(S(n)) \vdash \lnot \exists n: \lnot sigma(n) \justif{\quad}{17), Rule: non-contradiction}\\
&19) \Gamma, \sigma(0), \forall n: \sigma(n) \to \sigma(S(n)) \vdash \forall n: sigma(n) \justif{\quad}{18), Rule: $\lnot \exists$, Rule: Double Negation}
\end{alignat*}

\end{document}


